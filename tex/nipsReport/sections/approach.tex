\section{Approach}
    % general probabilistic model 
        % object recognition
        % action selection

    % implementation
        % object recognition, features n stuff
        % action selection, sampling n stuff

    \subsection{Probabilistic graphical model}
        This model consists of an object recognition subgraph extended in time for modelling actions. This model is agnostic of the type of features used so long a matching error function is defined for computing the similarity of features of the same type. Actions are optimally selected based on the minimum expected entropy of the object predictions for each potential action.
            
        \subsubsection{Object recognition}
            The object recognition subgraph discriminatively predicts object and pose based on observed features. This model consists of $N$ objects $O \in \{o_1,o_2, ..., o_N\}$ in $I$ poses $P \in \{p_1,p_2, ..., p_I\}$ evidenced by a set of $M$ features $\set{F} = \left\{ f_1, ...,  f_M\right\}$. This model assumes features are conditionally independent given object and pose resulting in the graphical model shown in \figref{fig:objectRecognitionSubgraph}.

            \begin{figure}[h]
              \centering
              \begin{tikzpicture}
                % Define nodes
                \node[obs]                            (F) {$f$};
                \node[latent, above=of F, xshift=-1.5cm] (O) {$O$};
                \node[latent, above=of F, xshift=1.5cm]  (P) {$P$};

                % Connect the nodes
                \edge {P,O} {F};

                % Plates
                \plate {} {(F)} {$M$}
              \end{tikzpicture}
              \caption{Object-recognition graphical model.}
              \label{fig:objectRecognitionSubgraph}
            \end{figure}

            % The likelihood of a feature, $\prob{f|o,p}$ is a distribution derived from feature matching errors. 
            Each feature in the model has an associated type $j$ and a value or descriptor to compute a similarity or matching error $\cursive{E}^j(\cdot,\cdot)$ with respect to another feature of the same type.

            This model predicts object and pose based on a likelihood distribution, $\prob{f|o,p}$, derived from matching errors between observed feature values, $\set{F}_{obs}$ and the set of reference feature values of the model, $\set{F}$. The features of the model are selected as the set of all unique features from all objects and poses observed in an ideal setting. Given an observation, $\set{F}_{\text{obs}}$, the best matching error with respect to a feature in the model $f^j \in \set{F}$ is given by \eqref{eq:bestMatch}.
            \begin{equation}
                \label{eq:bestMatch}
                \cursive{E}(f^j) = \min_{f^j_{\text{obs}} \in \set{F}_{\text{obs}}}\cursive{E}^j(f^j,f^j_{\text{obs}})
            \end{equation}
            
            Training data consists of $R$ sets of observed features for each object-pose pair. The feature likelihood distribution is learned from the set of best matching errors for all training observations of a specific object-pose.
            \begin{equation}
                \prob{f|o,p} \sim \{\cursive{E}_1(f), ...,  \cursive{E}_R(f)\}
            \end{equation}
            Note that this procedure works for any type of feature given an error function. This model can also combine multiple feature types into a single object recognition model. 
            
            The posterior for predicting object-pose is given by \eqref{eq:firstPosterior} assuming some prior, $\prob{o,p}$.
            \begin{align}
                \label{eq:firstPosterior}
                \prob{o,p|\set{F}} &= \frac{\prob{o,p} \cdot \prob{\set{F}|o,p}}{\prob{\set{F}}}\\
                \label{eq:setFeatureLikelihood}
                \prob{\set{F}|o,p} &= \prod_{m=1}^{M} \prob{f_m|o,p}\\
                \label{eq:setFeatureEvidence}
                \prob{\set{F}} &= \sum_{n,i} \prob{\set{F}|o_n,p_i}\cdot \prob{o_n,p_i}
            \end{align}

        \subsubsection{Action selection}
            To model actions, the object-recognition subgraph is extended into a time-series graph. Actions are modeled as $I$ pairwise relative pose transformations including the \italic{stay} action to determine when the algorithm has converged. The next pose $P_{t+1}$ is dependent only on the previous pose $P_t$ and the previous action $A_t$. This results in a graphical model shown in \figref{fig:fullGraph}.
            
            \begin{figure}[h]
                \centering
                \begin{tikzpicture}
                    % Define nodes
                    \node[obs] (F) {$f_1$};
                    \node[latent, above=of F]  (P) {$P_1$};
                    \node[latent, left=of P, xshift=-1.5cm] (O) {$O$};
                    \node[obs, above=of P]  (A) {$A_1$};

                    \node[obs, right=of F, xshift=1.5cm] (F2) {$f_2$};
                    \node[latent, above=of F2]  (P2) {$P_2$};
                    \node[obs, above=of P2]  (A2) {$A_2$};

                    \node[latent, right=of F2, xshift=1.5cm] (F3) {$f_{t+1}$};
                    \node[latent, above=of F3]  (P3) {$P_3$};
                    \node[latent, above=of P3]  (A3) {$A_3$};


                    % Connect the nodes
                    \edge {P,O} {F};
                    \edge {P2,O} {F2};
                    \edge {P3,O} {F3};
                    \edge {P,A} {P2};
                    \edge {P2,A2} {P3};

                    % Plates

                    \plate {pf} {(F)} {$M$}
                    \plate {pf2} {(F2)} {$M$}
                    \plate {pf3} {(F3)} {$M$}

                    \plate {} {(F)(P)(A)(pf)} {$t=1$}
                    \plate {} {(F2)(A2)(pf2)} {$t=2$}
                    \plate {} {(F3)(A3)(pf3)} {$t+1$}

                    % \draw [dotted, thick, scale=0.6] (7.5,1.5) -- (7.85,1.5);
                \end{tikzpicture}
                \caption{Full graph for object recognition with actions.}
                \label{fig:fullGraph}
            \end{figure}
            
            The posterior at time $t+1$ given the entire history of observations and actions is a Bayesian update of the posterior at time $t$ given in \eqref{eq:fullPosterior}.
            % {\small
            \begin{equation}
                \label{eq:fullPosterior}
                \prob{o,P_{t+1}|\set{F}_{1:t+1},A_{1:t}} = \frac{ \sum_{P_t} \prob{o,P_t|\set{F}_{1:t},A_{1:t-1}} \prob{\set{F}_{t+1}|o,P_{t+1}} \prob{P_{t+1}|P_t,A_t}}{\sum_{P_t,P_{t+1},O} \prob{O,P_t|\set{F}_{1:t},A_{1:t-1}} \prob{\set{F}_{t+1}|O,P_{t+1}} \prob{P_{t+1}|P_t,A_t}}
            \end{equation}
            % }
            
            We define the optimal action for object recognition as moving the object into the least ambiguous pose. This results in a minimum entropy of the distribution of posterior object prediction probabilities, $\prob{O|\set{F}_{1:t+1},A_{1:t}}$

            Because we have not yet observed $\set{F}_{t+1}$, we must compute the \italic{expected} entropy of the posterior in \eqref{eq:fullPosterior}. The optimal action is selected as the action which minimizes the expected entropy of object prediction posteriors across all potential actions defined by \eqref{eq:optimalAction}.
            % {\small
            \begin{equation}
                \label{eq:optimalAction}
                a^* = \argmin_{A_t} \expectedValue{ \entropy{O|\set{F}_{1:t+1},A_{1:t}} }{\set{F}_{t+1} \sim \prob{\set{F_{t+1}}|\set{F}_{1:t},A_{1:t}}}
            \end{equation}
            % }
            
    \subsection{Implementation}

        Our implementation of this model leverages some assumptions for more efficient computation with dynamic programming.

        \subsubsection{Object recognition}

            For object recognition, we used only SIFT features. We found that a normal distribution was sufficient for describing the distribution of best SIFT feature matching errors.
            \begin{equation}
                \prob{f|o,p} \sim \cursive{N}(\mu,\sigma)
            \end{equation}

            Assuming a uniform prior on objects and poses, the posterior simplifies from \eqref{eq:firstPosterior} to \eqref{eq:firstPosteriorComputation} which can be efficiently computed with dynamic programming.
            \begin{equation}
                \label{eq:firstPosteriorComputation}
                \prob{o,p|\set{F}} = \frac{\prob{\set{F}|o,p}}{\sum_{n,i} \prob{\set{F}|o_n,p_i}}
            \end{equation}

            In our experiment, we consider actions to be perfect, i.e. $\prob{P_{t+1}|P_t,A_t} \in \{ 0 , 1 \}$. Thus, given an action and a pose, the next pose can be computed deterministically. This simplifies our Bayesian update from \eqref{eq:fullPosterior} to \eqref{eq:fullPosteriorSimplified}.
            % {\small
            \begin{equation}
                \label{eq:fullPosteriorSimplified}
                \prob{o,P_{t+1}|\set{F}_{1:t+1},A_{1:t}} = \frac{\prob{o,P_t|\set{F}_{1:t},A_{1:t-1}} \prob{\set{F}_{t+1}|o,P_{t+1}} }{\sum_{P_t,O} \prob{O,P_t|\set{F}_{1:t},A_{1:t-1}} \prob{\set{F}_{t+1}|O,P_{t+1}} }
            \end{equation}
            % }

            This algorithm follows a posterior update procedure given in \eqref{eq:posteriorUpdateGeneral} and \eqref{eq:posteriorUpdateComputation} which is efficiently computed using dynamic programming by caching the values given in \tabref{tab:dynamicProgramming}.

            \begin{align}
                \label{eq:posteriorUpdateGeneral}
                \text{posterior}_t &= \frac{\text{posterior}_{t-1}*\text{likelihood}_t}{\text{evidence}_t}\\
                \label{eq:posteriorUpdateComputation}
                &= \frac{\text{posterior}_{t-1}*\text{likelihood}_t}{\sum \text{posterior}_{t-1}*\text{likelihood}_t}
            \end{align}

            \setlength{\tabcolsep}{.25em}
            \begin{table}[h]
                \caption{Cache for dynamic programming.}
                \label{tab:dynamicProgramming}
                \begin{center}
                \begin{tabular}{cccc} % number of columns and vertical lines
                    {\bf TIME} & {\bf POSTERIOR} & {\bf EVIDENCE} & {\bf LIKELIHOOD}
                    \\ \hline \\
                    0 & $\prob{o,p}$ & & \\[0.5em]
                    1 & $\prob{o,p|\set{F}_1}$ & $\prob{\set{F}_1}$ & $\prob{\set{F}_1|o,p}$ \\[0.5em]
                    2 & $\prob{o,p|\set{F}_1,\set{F}_2,A_1}$ & $\prob{\set{F}_2|\set{F}_1,A_1}$ & $\prob{\set{F}_2|o,p}$ \\[0.5em]
                    ...&...&...&...\\[0.5em]
                    t & $\prob{o,p|\set{F}_{1:t},A_{1:t-1}}$ & $\prob{\set{F}_t|\set{F}_{1:t-1},A_{1:t-1}}$ & $\prob{\set{F}_t|o,p}$ \\
                \end{tabular}
                \end{center}
            \end{table}

            \subsubsection{Action selection}

                To efficiently compute the optimal action given in \eqref{eq:optimalAction}, the posterior distribution is sampled using a Monte Carlo method. 

                First, the evidence distribution given in \eqref{eq:evidenceDistribution} needs to be sampled for each potential action, $A_{t}$.

                % {\small
                \begin{equation}
                    \label{eq:evidenceDistribution}
                    \prob{\set{F}_{t+1}|\set{F}_{1:t}, A_{1:t}} = \sum_{P_t,P_{t+1},O} \prob{O,P_t|\set{F}_{1:t},A_{1:t-1}} \prob{\set{F}_{t+1}|O,P_{t+1}} \prob{P_{t+1}|P_t,A_t}
                \end{equation}
                % }

                Because actions are deterministic, \eqref{eq:evidenceDistribution} simplfies to \eqref{eq:evidenceDistributionSimplified}.

                % {\small
                \begin{equation}
                    \label{eq:evidenceDistributionSimplified}
                    \prob{\set{F}_{t+1}|\set{F}_{1:t}, A_{1:t}} = \sum_{P_t,O} \prob{O,P_t|\set{F}_{1:t},A_{1:t-1}} \prob{\set{F}_{t+1}|O,P_{t+1}}
                \end{equation}
                % }

                This distribution can be sampled trivially by first sampling object-poses based on the discrete distribution defined by the previous posterior, $\prob{O,P_t|\set{F}_{1:t},A_{1:t-1}}$. Then, for each sampled object-pose, a particle representing a potential next observation is sampled from the feature likelihood distribution $\prob{\set{F}_{t+1}|O,P_{t+1}}$. Note that $P_{t+1}$ is computed deterministically.

                The next posterior is computed for each particle by \eqref{eq:fullPosteriorSimplified}. The posterior object probability is computed by marginalizing all poses given in \eqref{eq:objectPosterior}.

                \begin{equation}
                    \label{eq:objectPosterior}
                    \prob{O|\set{F}_{1:t+1},A_{1:t}} = \sum_{P_{t+1}} \prob{O,P_{t+1}|\set{F}_{1:t+1},A_{1:t}}
                \end{equation}

                The entropy of the posterior object probabilities is computed for each particle and then averaged to give the expected entropy for each potential action. The optimal action is selected according to \eqref{eq:optimalAction} by the action which has the minimum expected entropy which leads to a pose with the most discriminative features of the most likelily objects.
