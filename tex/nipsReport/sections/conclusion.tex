\section{Conclusions}
    
    This paper shows that static object recognition is limited for certain robotics problems. This is emphasized by 3 difficult object recognition tasks which motivates the need for interacting with the environment. 

    A probabilistic graphical model was designed to recognize objects and poses based on any type of learned features. This model was extended to incorporate actions over time. A solution was derived for determining the optimal action for improving object recognition, and an experiment on the ambiguous book problem shows promising results.

    There are several areas for future work in this domain. First is with the static object recognition subgraph. Currently, we are facing an issue with overfitting when we train the ambigious poses with different sets of data. This has been troubling and we set this issue aside by training the ambiguous poses on the same set of data. We believe that this may be an issue with SIFT features and suspect that incorporating other features into this model such as 3D point cloud or color histogram features may help with this overfitting problem.

    Other future work lies in loosing our constraints on discrete poses with perfect actions into continuous poses and continuous actions. %Relative perspective can also be considered - not only can the robot rotate the object, but the robot can also just move around the object.

    For a practical application, this algorithm would need to perform in a cluttered environment, observing many objects at the same time, some of which may have never been seen before. Future work could include searching for objects of interest and inspecting potential objects until the correct one is found.

